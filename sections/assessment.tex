\begin{description}

  \item \textsc{Class participation (20–60\%):} The success of any course is
    directly related to the levels of engagement brought both by the instructor
    and the students. As such, class participation is vitally important.
    Similarly, though attendance is logically required for class participation,
    it is not sufficient. This class requires active participation both inside
    the classroom and outside. No “passive consumers,” as a professor of mine
    put it. 
    
    You can miss up to three meetings without penalty, and you can use these
    opportunities tactically, to provide space and time to either fulfill
    other obligations or recuperate from the previous night. I don’t care why
    you didn’t come. I start to care with the fourth absence, and I start to
    require documentation. Repeated unexcused absence quickly gobbles up the
    class participation component of the grade and begins to threaten your
    ability to even \textit{pass} the course.

    Because this course is discussion-oriented, active participation means,
    most importantly, participating in the discussions in class. But useful and
    engaged participation in discussions also depends on good preparation,
    which includes doing the reading for the course. I encourage (but will not
    collect) you to think of one or two points of entry into a discussion of a
    text for each meeting. This could be a point of confusion (don’t be shy!),
    a point of comparison/contrast between passages to another work, or a
    useful parallel to something outside the coursework. Come to class with
    questions, in other words, and writing them out as mini-prompts may be
    especially helpful.

  \item \textsc{Dérive project (20–60\%):} You will undertake two dérives during
    the semester. The first will be at the end of the first unit, and the
    second will be at the end of the second unit. In the first, your starting
    point will be assigned completely at random. The second will begin with a
    randomly chosen point from one of the novels we have read. In both, you
    will get lost in Manhattan (and beyond?), while also documenting and
    tracking yourself.
    
    In order to direct your dérive, you will use either an app for your
    smartphone or a set of cards printed out ahead of time. In order to track
    the dérive, you are required to trace your path and take notes on a Field
    Papers atlas and, if possible, track yourself using GPS.

    During the course of the dérive, which can last hours, you should reflect
    on the readings we have already done for the class, both in what you
    observe while getting lost, but also in the process of getting lost itself.
    
    At the end, you will write up a short (1,100--1,250 words) report in
    Markdown for each dérive, including textual references from our readings.
    The report will be joined by a polyline shapefile of your journey and the
    original Field Papers atlas with notes. You can and are encouraged to use
    other forms of media to supplement the report.

  \item \textsc{\texttt{NYWalker} (20–60\%):} This class, similarly to a class
    being co-taught this semester by myself and Prof.\ Thomas Augst, will be
    contributing to a geospatial data collection forked from the one dedicated
    to New York novels known as “\texttt{NYWalker}”
    (\texttt{http://github.com/nyscapes/nywalker/}).  All four novels we read
    will be geocoded by hand, with the last three done by you and your
    classmates without substantial direction from me. Each of you will be
    assigned a portion of one of the novels, and then it will be your
    responsibility to populate the \texttt{NYWalker} database.

    Once you are done geocoding, you will submit a very short (700--800
    words) report of the process, including examples of difficulties faced and
    suggestions for improvements.

  \item \textsc{Final project (20–60\%):} For a final project, you will create a
    digital story about one or two of the novels we read in class that will
    incorporate both textual interpretation informed by the critical readings
    we have done this semester as well as an original geospatial analysis of
    places mentioned in the novel, as revealed by our geospatial database. The
    goal is to tie together both methods of approaching a novel and seeing how
    the different methods complement and resonate with each other.

    The story itself will be an interactive version of a long paper
    (2,500--3,000 words) that forms the intellectual backbone of the story.
    This paper should argue a specific point, using both “traditional” literary
    modes of analysis (such as close reading or symptomatic reading) and
    geospatial digital analysis to persuasive effect.

    Both the story and the paper should follow standard scholarly guidlines
    regarding citation.

\end{description}
