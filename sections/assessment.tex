\begin{description}

  \item \textsc{Class participation (16–26\%):} The success of any course is
    directly related to the levels of engagement brought both by the instructor
    and the students. As such, class participation is vitally important.
    Similarly, though attendance is logically required for class participation,
    it is not sufficient. This class requires active participation both inside
    the classroom and outside. No “passive consumers,” as a professor of mine
    put it. 
    
    You can miss up to three meetings without penalty, and you can use these
    opportunities tactically, to provide space and time to either fulfill
    other obligations or recuperate from the previous night. I don’t care why
    you didn’t come. I start to care with the fourth absence, and I start to
    require documentation. Repeated unexcused absence quickly gobbles up the
    class participation component of the grade and begins to threaten your
    ability to even \textit{pass} the course.

    Because this course is discussion-oriented, active participation means,
    most importantly, participating in the discussions in class. But useful and
    engaged participation in discussions also depends on good preparation,
    which includes doing the reading for the course. I encourage you to think
    of one or two points of entry into a discussion of a text for each meeting.
    This could be a point of confusion (don’t be shy!), a point of
    comparison/contrast between passages to another work, or a useful parallel
    to something outside the coursework. Come to class with questions, in other
    words, and writing them out as mini-prompts may be especially helpful.

  \item \textsc{Critical presentation (16–26\%):} While reading the critical
    history of \textit{The Great Gatsby}, each of you will give a short (8–10
    minutes) presentation that introduces either one of the essays cited in the
    reading for that day or an essay you have found via JSTOR contemporary with
    that day’s reading. You will get to sign up for a time period on the first
    day of class. The presentation will also have a short (1 page) written
    component that you will turn in. The goal of the presentation is to
    introduce new knowledge to the class that it has not already had, thereby
    facilitating that day’s discussion. This presentation cannot make use of
    the computer in the classroom, and you should email the subject of your
    presentation to me 24 hours in advance.

  \item \textsc{Digital presentation (16–26\%):} During the digital section of
    the course, each of you will give a short (8–10 minutes) presentation on a
    digital tool or project you have found online, possibly even making some
    quick use of the \textit{Gatsby} dataset. The tool or project should be
    appropriate to that day’s work, meaning statistical analysis tools and
    projects fit for the \texttt{Python} days, while spatial analysis tools and
    projects fit better for the \texttt{Carto} days. You can sign up on the
    first day for these presentations. Your presentation will also have a short
    (1 page) written component that you will turn in. The goal of the
    presentation is to introduce a new form of digital reading to the class
    that may spark new ideas for final projects. This presentation can make use
    of the computer in the classroom, and you should email the subject of your
    presentation to me 24 hours in advance.

  \item \textsc{Novel presentation (16–26\%):} While reading the two more
    “recent” notels, each of you will give a short (8–10 minutes) presentation
    that introduces secondary materials to the novel at hand, be it scholarly
    writing found via JSTOR or popular writing found online or in other
    publications. You will get to sign up for a day on the first day of class.
    The presentation will also have a short (1 page) written component that you
    will turn in. The goal of the presentation is to introduce new sources to
    the class that it has not already had, thereby facilitating that day’s
    discussion. This presentation cannot make use of the computer in the
    classroom, and you should email the subject of your presentation to me 24
    hours in advance.

  \item \textsc{Final project (26–36\%):} The final project is an extension of
    the work you have been doing all semester. It has two obvious components,
    as do the previous assignments, both an in-class presentation and a written
    component. The goal is for you to synthesize the different methods of
    reading we have learned over the semester to provide the foundation of a
    new, mixed reading of one of the semester’s novels. As such, you are
    encouraged to make use of the digital tools (and expand on your knowledge
    of them) while also keeping in mind the various secondary materials to
    which you have been exposed.  Remember the question guiding this
    course—“does it work?”—when considering how to develop the argument of your
    project. 

    The presentation, for which you will sign up on the first day, will be
    longer (18–20 minutes) than the previous presentations, and there will be
    time for Q\&A with your classmates afterward. You will be able to make use
    of the computer in the classroom to visualize or guide your findings.

    In addition to the presentation, you will submit a longer (7 pages) written
    component that provides a more in-depth view of your project.

    You are \textit{strongly} encouraged to be thinking about the project all
    semester long, and you should have a good sense of at least the topic by
    the time we return after Thanksgiving. Making use of office hours is a
    great way to help your project take shape in the rushed final weeks of the
    semester.

    Both the presentation and the write-up should follow standard scholarly
    guidlines regarding citation.

\end{description}
