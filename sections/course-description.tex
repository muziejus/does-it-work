Over the past decade, literary study has become increasingly, reflexively
interested in investigating the methods that it generates.  At the same time,
the digital humanities have emerged to claim a part of growing English
departments. This course serves as an introduction to both of these currents in
contemporary literary study. We will consider both familiar forms of reading as
well as new, different forms that have broadened the way we interpret texts,
and we will then look to how the digital humanities, as an especially
method-oriented subfield, brings these questions of interpretation into even
sharper contrast. Students will, then, see how debates about interpretation are
lived and experienced within the digital humanities, both in theory and
practice. Finally, students will learn to use digital methodologies in their
interpretation, with possible projects in corpus analysis, topic modeling, and
geospatial analysis.
