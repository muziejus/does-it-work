What is the relationship between the city and the novel? A century ago, the
pioneers of American sociology in Chicago often looked to literature for
inspiration. Later authors returned the favor in showing sociological
influences in their own urban imaginations. This class returns to this old push
and pull with a distinct, contemporary spin. Three units — one on reading about
the city, one on reading novels about the city, and one on making maps of the
city as revealed in the novels — make up the semester and push us toward our
goal of visualizing what the relationships between city and novel were, are,
and could be. The maps we make will be digital, but no previous experience in
programming or mapping is required. Students will learn everything during the
semester, and they will be able to make their own maps and draw spatial
conclusions about textual works for their final projects. The novels we read
will be American works of the twentieth and twenty-first centuries, and they
will focus geographically on the U.S. as well. The early sociological texts
will be supplemented by more contemporary writings on the city as well as texts
from the digital mapping debates in the field of geography. 
