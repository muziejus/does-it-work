\begin{itemize}

	\item to introduce you to general concepts bridging sociology and literary
criticism through the objects of the city and the novel as well as concepts
pertaining to the geospatial digital humanities and qualitative GIS;

  \item to develop skills in 

    \begin{itemize}

			\item reading analytical and literary texts as well as geographic data
sources and maps,

			\item writing analyses that are cogent and syncretic, making use of the
various methods on hand, and

			\item creating (as well as using and distributing) geospatial datasets,
GISes, and cartographic visualizations of the former; and

    \end{itemize}

	\item to develop, refine, and present scholarship that exists, spatially and
temporally, beyond the boundaries of the course.

\end{itemize}

