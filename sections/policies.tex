\begin{description}

  \item \textsc{Assignments:} The assignment instructions, though detailed in
    the syllabus, may be enhanced or supplemented during the course. If you
    have any questions about an assignment, you should ask for clarification
    early. The assignments are due on the dates noted in the syllabus. 
    
    Nearly everything can be submitted electronically. Shapefiles, Markdown
    texts of dérives, and digital stories, of course, must be. On the other
    hand, anything with a word-count can additionally be submitted during
    class, printed out. I don’t want to deny you the pleasure of dropping a
    physical manifestation of your hard work on a hard surface at the start of
    class. The Field Papers atlas from each dérive absolutely must be submitted
    as a physical object. 
    
    Late assignments jeopardize both your and my rhythms in the class, so they
    will be penalized. I will give you feedback and will happily discuss any
    work with you, but grades should be considered final.

    Additionally, grading is variable based on what you feel your strengths are.
    Each assignment will be worth at least 20\% of your final grade, but the upper
    limit of the grade is set by you. You should email me how you slice up the
    pie by the end of the second week.

  \item \textsc{Attendance:} As indicated above, attendance is required. Three
    absences will be excused without supplemental documentation, and I
    encourage you to use these tactically. Catching up is your responsibility.

    Subsequent absence requires formal documentation. Otherwise it begins to
    harm your final grade. Though class participation is only part of the final
    grade, extreme absenteeism (more than six meetings missed) may put your
    ability to pass the course at risk.

    Please show up on time to class, as well.

  \item \textsc{Digital learning:} I, along with my colleagues at NYU Data
    Services, will be exposing you to a lot of new tools and concepts. Our
    class will occasionally meet in Bobst Library, where we will learn how to
    use ArcGIS and how to create digital stories that incorporate cartographic
    elements. These skills are difficult, and both I and the team at Data
    Services will try to help as much as possible. But please aim to attend
    each of the teaching classes at the library, in order to gain familiarity
    with the technology that will come in handy later.

    In addition to what Data Services offers, during the course of the class we
    will also learn how to write digital documents in Markdown, complete with
    embedded multimedia objects. We will also learn how to use a gazetteer as
    part of the \texttt{NYWalker} project.

    None of this requires special previous knowledge.

  \item \textsc{Electronics:} Despite the presence of the digital, especially
    as the class gets deeper into the semester, our time in class is meant as a
    sanctuary from the distractions of the rest of the world. Furthermore, the
    class relies on discussion and engagement, and the front of a laptop screen
    is a brilliant shield behind which a student can hide, even
    unintentionally. During our meetings, then, there can be no use of
    electronic devices. Please also set whatever devices you have but aren’t
    using to silent mode.

  \item \textsc{Course website:} A public website, mostly including this
    syllabus but with richer bibliographic detail and assignment descriptions,
    is available at \url{\mycourseurl}.

  \item \textsc{Communication:} Communication is vitally important to the
    pedagogical process, and this course depends on clear communication in both
    directions. If you have questions, comments, or concerns, the best course
    of action is to come visit me during my office hours (\myofficehours). If
    your questions, etc., cannot wait until then, then clearly you can also
    email me at \texttt{\myemail}. I should respond within 48 hours.

    This is a new course, meaning that there will be even more unfinished edges
    ready to scratch someone than in a typical course. We have a collective
    goal of learning, however, so if the unfinished edges get to be
    overwhelming, I’ll adjust the parameters of the course appropriately. I’m
    not out to catch you, nor is this course a process of grotesque punishment.
    Please don’t treat it as such.

    Once more, with feeling: \textit{communication is vitally important to the
    pedagogical process}. If you have concerns or worries, please let me know
    about them sooner rather than later.

  \item \textsc{Disabilities:} If you have a disability, you should register
    with the Moses Center for Students with Disabilities (mosescsd@nyu.edu; 726
    Broadway, 2nd Floor, 212.998.4980), which can arrange for things like extra
    time for assignments. Please inform me \textit{at the beginning of the
      semester} if you need any special accommodations regarding the
    assignments.

  \item \textsc{Academic integrity:} Please look at NYU’s full statement on
    academic integrity, available at \url{http://cas.nyu.edu/page/academicintegrity}.
    Any instance of academic dishonesty will result in an F and will be
    reported to the relevant dean for disciplinary action. Remember that
    plagiarism is a matter of fact, not intention. Know what it is, and don’t
    do it.

  \item \textsc{This document:} This source code and documentation for this
    syllabus is available at
    \url{https://github.com/muziejus/the-map-of-the-city-that-is-this-novel}.
    The syllabus is \copyright\ 2015, \myauthor. It is licensed as Creative
    Commons 3.0 \textsc{by-nc-sa}, giving you permission to share and alter it
    in any way, as long as it is for non-commercial purposes, maintains the
    license, and gives proper attribution. Further information regarding the
    license, the history of the document, and influences can be viewed
    at the Github repository.

\end{description}
