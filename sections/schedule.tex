Readings that are not the four books listed above will be available via the
course website. See the list of references at the end for details.

\subsection*{1. That Old \textit{Gatsby}, That Critique \textit{Gatsby}}

  In the first section of this course, we’ll be returning to a familiar,
  canonical work of American 20th century literature, \textit{The Great
    Gatsby}. Next, we will follow our own reading of the novel with a look at
  the novel’s critical history. 

\begin{enumerate}

  \item Monday, 5 Sep: \textbf{No Class.}

    Wednesday, 7 Sep: Introductions and a snippet from Massumi’s “Translator’s Foreword: Pleasures of Philosophy.”

  \item Monday, 12 Sep: Fitzgerald, \textit{The Great Gatsby}, 

    Wednesday, 14 Sep: Fitzgerald, \textit{The Great Gatsby},

  \item Monday, 19 Sep: Fitzgerald, \textit{The Great Gatsby},

    Wednesday, 21 Sep: \textbf{\small Critical presentations begin}, Tredell, intro \& ch. 1.

  \item Monday, 26 Sep: Tredell, chs. 2 \& 3.

    Wednesday, 28 Sep: Tredell, chs. 4 \& 5.

\end{enumerate}

\subsection*{2. A Theoretical Break}

  Next, we take a short break to learn about the stresses offered by these more critique-driven forms of reading.

\begin{enumerate}
  \setcounter{enumi}{4}
 
  \item Monday, 3 Oct: Bersani, “Pynchon, Paranoia, and Literature” and Latour,
    “Why Has Critique Run out of Steam? From Matters of Fact to Matters of
    Concern.”

    Wednesday, 5 Oct: Sedgwick, “Paranoid Reading and Reparative Reading, or
    You’re So Paranoid, You Probably Think This Essay Is about You” and
    selections from Felski, \textit{The Limits of Critique}.
  
\end{enumerate}

\subsection*{3. The New \textit{Gatsby}, The Digital \textit{Gatsby}}

  These five weeks serve as an opportunity to learn new methods of literary
  criticism, now based in digital tools. We will learn how to use
  \texttt{Voyant} to quickly see patterns in the text of \textit{The Great
    Gatsby}, \texttt{R} to analyze the text within a high-level statistical
  programming environment,\footnote{The unit on \texttt{R} is adapted from
    Jockers.} and, finally, \texttt{NYWalker} and \texttt{Carto} to learn how
  to make maps to analyse the geographical space of \textit{The Great Gatsby}.

\begin{enumerate}
  \setcounter{enumi}{5}
 
  \item Monday, 10 Oct: \textbf{No Class.}

    Wednesday, 12 Oct: \textbf{\small Digital presentations begin}, Ramsay,
    “Algorithmic Criticism” and \texttt{Voyant} and \textit{The Great Gatsby}
  
  \item Monday, 17 Oct: Introduction to \texttt{R}

    Wednesday, 19 Oct: \texttt{R} and \textit{The Great Gatsby}

  \item Monday, 24 Oct: \texttt{R} and \textit{The Great Gatsby}

    Wednesday,  26 Oct: \texttt{R} and \textit{The Great Gatsby}

  \item Monday, 31 Oct: \texttt{NYWalker} and \textit{The Great Gatsby}

    Wednesday, 2 Nov: \texttt{Carto} and \textit{The Great Gatsby}

  \item Monday, 7 Nov: \texttt{Carto} and \textit{The Great Gatsby}

    Wednesday, 9 Nov: \texttt{Carto} and \textit{The Great Gatsby}

\end{enumerate}

\subsection*{4. The New Novels, The New Systems}

  The semester closes with reading two new(er) novels that invite a systematic,
  totalized reading. We close with student presentations on their final
  projects.

\begin{enumerate}
  \setcounter{enumi}{10}

  \item Monday, 14 Nov: \textbf{\small Novel presentations begin}, Pynchon, chs. 1 -- 3.
  
    Wednesday, 16 Nov: Pynchon, chs. 4 \& 5.

  \item Monday, 21 Nov: Pynchon, ch. 6.

    Wednesday, 23 Nov: \textbf{No Class.}

  \item Monday, 28 Nov: Spiotta, pts. 1 \& 2.

  Wednesday, 30 Nov: Spiotta, pts. 3 \& 4.

  \item Monday, 5 Dec: Spiotta, pts. 5 -- 7.

  Wednesday, 7 Dec: Spiotta, pts. 8 \& 9.

  \item Monday, 12 Dec: \textbf{\small Final presentations}
  
  Wednesday, 14 Dec: \textbf{\small Final presentations}

\end{enumerate}
