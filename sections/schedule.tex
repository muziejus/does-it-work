Readings that are not the four novels listed above will be available on reserve
or by other means. See the list of references at the end for details.

\subsection*{1. The organic, responsive city}

  This unit is made up almost entirely of scholarly work on the city in general
  from the fields of sociology, anthropology, and geography. Some texts,
  however, have a specific literary component to them as being influenced by,
  responses to, or examples of literary practice. 

\begin{enumerate}

  \item Tuesday, 26 Jan: Introduction, getting-to-know-you, fantasies and expectations,
    Whitehead, “City Limits”

    Thursday, 28 Jan: Park, “The City: Suggestions for the Investigation of Human Behavior in the Urban Environment” (p. 1--12!), Burgess, “The Growth of the City: An Introduction to a Research Project,” and Wright, “Introduction”

  \item Tuesday, 2 Feb: Harvey, “Henri Lefebvre’s Vision,” “The Right to the City,” “The Creation of the Urban Commons,” and “\#OWS: The Party of Wall Street Meets Its Nemesis”

    Thursday, 4 Feb: De Certeau, “‘Making Do’: Uses and Tactics,” “Walking in the City,” and “Spatial Stories”

  \item Tuesday, 9 Feb: Latour, “Introduction: How to Resume the Task of Tracing Associations,” and “On the Difficulty of Being an ANT: An Interlude in the Form of a Dialog”

    Thursday, 11 Feb: \texttt{GPS, Markdown, dérive omnibus} training day. Debord, “Definitions,” “Introduction to a Critique of Urban Geography,” and “Theory of the Dérive”

\end{enumerate}

\subsection*{2. The city in the novel}

Having read different ways we can consider the city as a process, bubbling with
potential, it’s now time to see how (or if) these novels play with what
emanates form that view of the city.

\begin{enumerate}
 \setcounter{enumi}{3}
\item Tuesday, 16 Feb: \texttt{ArcGIS} training day.

  Thursday, 18 Feb: Wharton, \textit{The Age of Innocence}, chs. 1--11

  \item Tuesday, 23 Feb: Wharton, \textit{The Age of Innocence},  chs. 12--24

    Thursday, 25 Feb: \textbf{\small Dérive 1 due}, Wharton,  \textit{The Age of Innocence}, chs. 25--end
  
  \item Tuesday, 1 Mar: Dos Passos, \textit{Manhattan Transfer}, “Ferryslip”--“Tracks”

    Thursday, 3 Mar: Dos Passos,  \textit{Manhattan Transfer}, “Steamroller”--“Longlegged Jack of the Isthmus”

  \item Tuesday, 8 Mar: Dos Passos,  \textit{Manhattan Transfer}, “Nine Days Wonder”--“Nickelodeon”

    Thursday, 10 Mar: Dos Passos,  \textit{Manhattan Transfer}, “Revolving Doors”--end

  \item \textbf{Spring Break, no class}

  \item Tuesday, 22 Mar: Cole, \textit{Open City}, chs. 1--15

    Thursday, 24 Mar: Cole,  \textit{Open City}, chs. 16--end

  \item Tuesday, 29 Mar: Kushner, \textit{The Flamethrowers}, chs. 1--4

    Thursday, 31 Mar: Kushner, \textit{The Flamethrowers}, chs. 5--9

  \item Tuesday, 5 Apr: Kushner, \textit{The Flamethrowers}, chs. 10--13 
  
    Thursday, 7 Apr: Kushner, \textit{The Flamethrowers}, chs. 14 \& 15

  \item Tuesday, 12 Apr: Kushner, \textit{The Flamethrowers}, chs. 16--end 

    Thursday, 14 Apr: \texttt{Geospatial analysis} training day.

\end{enumerate}

\subsection{3. Qualitative GIS}

As we wind down the semester, our thoughts turn to how we can synthesize our
different approaches to the text. We’ve read social scientific takes on the
city followed by novels that are engaged with New York City (and other
places!). In the meantime, we’ve been collecting a pile of geographic data
about these novels and don’t yet really know what to do with it. Now, with a
few caveats, we’ll try to figure out new paths for analysis.

\begin{enumerate}
  \setcounter{enumi}{12}

  \item Tuesday, 19 Apr: \textbf{\small Dérive 2 due}, \texttt{Esri Story Maps} training day.

    Thursday, 21 Apr: Haraway, “Situated Knowledges: The Science Question in Feminism and the Privilege of Partial Perspective”

  \item Tuesday, 26 Apr: \textbf{\texttt{NYWalker} \small report due}, Woodward, Jones, and Marston, “Of Eagles and Flies: Orientations toward the Site,” Gibson-Graham, “Diverse Economies: Performative Practices for ‘Other Worlds’”

    Thursday, 28 Apr: Elwood and Cope, “Introduction: Qualitative GIS: Forging Mixed Methods through Representations, Analytical Innovations, and Conceptual Engagements,”  Pavlovskaya, “Non-quantitative GIS”

  \item Tuesday, 3 May: Liu, “Where Is Cultural Criticism in the Digital Humanities?” Risam, “Beyond the Margins: Intersectionality and the Digital Humanities”

    Thursday, 5 May: Class wrap-up and review

  \item Thursday, 12 May: \textbf{\small Digital story due}

\end{enumerate}
